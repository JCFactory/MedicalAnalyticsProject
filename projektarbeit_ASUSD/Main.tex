\input{preamble}

\newacronym{lda}{LDA}{Latent Dirichlet Allocation}
\newacronym{ngs}{NGS}{Next Generation Sequencing}
\newacronym{ncbi}{NCBI}{National Center for Biotechnology Information}
\newacronym{all}{ALL}{Acute Lymphoblastic Leukemia}
\newacronym{msa}{MSA}{Multiple Sequence Alignment}
\newacronym{muscle}{MUSCLE}{Multiple Sequence Comparison by Log- Expectation}
\newacronym{dna}{DNA}{Desoxyribonucleic acid}
\newacronym{ir}{IR}{Information Retrieval}
\newacronym{ner}{NER}{entity recognition}
\newacronym{ie}{IE}{Information extraction}
\newacronym{kd}{KD}{Knowledge Discovery}
\newacronym{nlp}{NLP}{Natural Language Processing}
\newacronym{rna}{RNA}{Ribonucleic Acid}
\newacronym{cart}{CART}{Classification and Regression Tree}
\newacronym{idb}{IDB}{Iterative Dichotomized 3}
\newacronym{dt}{DT}{Decision Tree}
\newacronym{pca}{PCA}{Principal Component Analysis}
\newacronym{lda}{LDA}{Linear Discriminant Analysis}
\newacronym{tfidf}{TFIDF}{term frequency/inverse document frequency}
\newacronym{rst}{RST}{Rhethorical Structure Theory}
\newacronym{hmm}{HMM}{Hidden-Markov-Models}
\newacronym{bm}{BM}{Bayesian models}




\begin{document}
		\begin{titlepage}
			\begin{center}
				\begin{spacing}{2}
					\includepdf{Image/FOM_Logo.pdf}
					\textbf{\large FOM - Hochschule für Oekonomie \& Management \\
						Hamburg \\
						\ \\
						Master-Studiengang Big Data \& Business Analytics \\
						x. Semester \\
						\ \\
						Titel\\
						Titel\\
						Titel 
						}
				\end{spacing}

				\vfill
				
				\begin{tabbing}
					\hspace{2.5cm}\=\kill
					Betreuer: \>  \\
					\> Dozent im Fach \glqq \grqq \\
					\ \\
					Autor: \> Vorname Nachname \\
					\> Straße \\
					\> PLZ Ort \\
					\> Matrikel-Nr:  \\
					\> x. Fachsemester \\
					\ \\
					Ort, den \today
				\end{tabbing} 
			\end{center}
		\end{titlepage}

%\includepdf{Image/Deckblatt.pdf}

			\setcounter{tocdepth}{3}
			\setcounter{secnumdepth}{3}		
			\pagenumbering{Roman}
			\thispagestyle{empty}
			\pdfbookmark{\contentsname}{toc}\tableofcontents
			\newpage
			\listoffigures
			\listoftables
			\printglossary[type=acronym,style=listdotted,title=Abkürzungsverzeichnis,toctitle=Abkürzungsverzeichnis] 
			\newpage
			\pagenumbering{arabic}
			\thispagestyle{empty}

\chapter{Abstract}\label{abstract}
\ref{abstract}

\chapter{Introduction}\label{introduction}

\chapter{Related work}\label{related}
analysis of genetic mutations which cause breastcancer \autocite{breastcancer}

\autocite{zhao_2016} et al. describe how topic modeling can be used to analyze \gls{ngs}. 

\chapter{\gls{lda}}\label{lda}
\section{General description}\label{lda_description}
\gls{lda} was developed by David Blei et. al in the year 2003 and is a clustering algorithm for text mining. It counts to the most popular topic modelling algorithms \autocite{zhao_2016}.
According to \autocite{zhao_2016}, topic modelling requires of a number documents which represent each of them a mixture of latent topics. Moreover, each topic is expressed by a distribution of words. During \gls{lda}, two relationships are analyzed: First, the relationship between documents and words, also called 'per-document topic distributions'. Second, the relationship between words and topics ('per-topic word distributions'). To measure the relationships exactly and to make inference about topics and documents for text mining, probability matrices are calculated.
 
\section{Examples and possible uses cases}\label{lda_examples}
images and schemes
\section{Python package 'Gensim'}\label{gensim}
              
\chapter{Acute Lymphoblastic Leukemia}\label{all}
\section{Types of Leukemia and its causes}\label{leukemia_types}
\section{Examples for Genome Analysis: \gls{ngs}}\label{genome_analysis}
\gls{ngs} refers to post-Sanger sequencing methods \autocite{zhao_2016}. Furthermore, since \gls{ngs} produces large volumes of sequence data it might be very useful to use topic modelling techniques to maintain the flexibility for the level of resolution required for given experiments.  
The step before analyzing two or more (multiple) genomes is called alignment which includes a comparison of two genomes. There are many different types of alignments, but Zhao et al. refer to the \gls{msa} by describing \gls{muscle} and \gls{clustal}.

But there is not only the \gls{ngs} technique to analyze, but also many other methods to analyze genomes as described by \autocite{zhao_2016}. 
Before analy

\section{Data sources: \gls{ncbi} and Ensembl genome browser 96}\label{datasources}

\chapter{Development of a solution for genetic analysis of \gls{all} genomes by implementing \gls{lda}}\label{development}
\section{Problems and challenges of genetic analysis}\label{problems_challenges}
\section{First steps: Draft of developed solution}\label{draft}

To get useful data, the \autocite{ncbi} was used to get all currently detected mutations of genomes which may cause \gls{lda}.

The first idea was to build a parsing application, which iterates over the found 582 genomes. After the iteration, it compares the oncogenes with the healthy genomes.
Eine spontane Idee, die mir gerade einfiel, wäre, eine Art Parser zu bauen, der über die 582 Gene iteriert und diese mit den gesunden Genen (ebenfalls aus einer Suchanfrage aus NCBI entnommen) zu vergleichen und somit die Unterschiede herauszufinden (und evt. in einem Diagramm darzustellen). Ggf. lassen sich die Unterschiede auch clustern und mit dem LDA-Verfahren kombinieren.
diagrams, schemes
\section{Proposed solution}\label{proposed_solution}
\section{Results}\label{results}

\chapter{Conclusion and Outlook}\label{conclusion_outlook}
\section{Lessons learned}\label{lessons_learned}
\section{Conclusion}\label{conclusion}
\section{Outlook}\label{outlook}	
	
% Beispiel für Bild
		\begin{figure}[htbp]
			\centering
			%\includegraphics[width=1\textwidth]{Image/xxx.pdf}
			\caption[xxx]{Bildunterschrift}
			\label{xxx}
		\end{figure}
		
		
% Beispiel Zitat		
		\begin{quote}
			\textit{Ein Zitat}
		\end{quote}


% Quellenangabe 		\autocite[Seite]{Autor.Jahr}
				\autocite[20]{}

\input{Verzeichnis}
\newpage

\input{Anhang}
\bibdata{Lit}



\end{document}
