\documentclass[12pt, a4paper]{scrreprt}

\renewcommand*\familydefault{\sfdefault} 
%\usepackage[T1]{fontenc}

\usepackage[ngerman]{babel}
\usepackage{cite}
\usepackage[utf8]{inputenc}
%\usepackage[onehalfspacing]{setspace}
\usepackage{geometry, textcomp}
\newgeometry{right=2cm,left=4cm, top=2.5cm, bottom=2.3cm, footnotesep=0.5cm}

\usepackage[acronym]{glossaries}	% creates a glossary
\glsenablehyper
\makeglossaries

\usepackage[automark,headsepline,ilines,komastyle]{scrpage2}
\usepackage{amsmath,amssymb,amstext}
\usepackage{blindtext}
\setlength{\parindent}{0pt}
\setlength{\headheight}{1.5\baselineskip}
\renewcommand{\baselinestretch}{1.5}

\pagestyle{scrheadings}
\clearscrheadfoot
\ihead[]{}
\chead[]{}
\ohead[]{\headmark \hfill \thepage}
\ifoot[]{}
\cfoot[]{}
\ofoot[]{}

\setheadsepline[\textwidth]{1pt}
\usepackage{tabularx}
\usepackage{colortbl}
\usepackage{multirow}
\usepackage{hhline}
\usepackage{array}
\usepackage{tocloft}
\usepackage[hidelinks]{hyperref}
\tocloftpagestyle{scrheadings}
\renewcommand{\chapterpagestyle}{scrheadings}
\usepackage[font=footnotesize]{caption}

\usepackage{tikz}
\usepackage{rotating} 

\newenvironment{packed_item}
	{\begin{itemize}
			\setlength{\itemsep}{0pt}
			\setlength{\topsep}{0pt}
			\setlength{\parsep}{0pt}
			\setlength{\parskip}{0pt}}
		{\end{itemize}}
	
\usepackage[style=authoryear, natbib=true, backend=biber]{biblatex}

\renewcommand{\nameyeardelim}{ }
\usepackage[babel,german=guillemets]{csquotes}

\makeatletter

\newrobustcmd*{\parentexttrack}[1]{%
	\begingroup
	\blx@blxinit
	\blx@setsfcodes
	\blx@bibopenparen#1\blx@bibcloseparen
	\endgroup}

\AtEveryCite{%
	\let\parentext=\parentexttrack%
	\let\bibopenparen=\bibopenbracket%
	\let\bibcloseparen=\bibclosebracket}

\makeatother

\usepackage{pstricks}
\usepackage{pstricks-add}


\addbibresource{Lit.bib}
\usepackage[final]{pdfpages}

\newacronym{tmg}{TMG}{Telemediengesetz}
\newacronym{bdsg}{BDSG}{Bundesdatenschutzgesetz}
\newacronym{uwg}{UWG}{Gesetz gegen unlauteren Wettbewerb}
\newacronym{seo}{SEO}{Search Engine Optimization}






\begin{document}
		\begin{titlepage}
			\begin{center}
				\begin{spacing}{2}
					\includepdf{Image/FOM_Logo.pdf}
					\textbf{\large FOM - Hochschule für Oekonomie \& Management \\
						Hamburg \\
						\ \\
						Master-Studiengang Big Data \& Business Analytics \\
						x. Semester \\
						\ \\
						Titel\\
						Titel\\
						Titel 
						}
				\end{spacing}

				\vfill
				
				\begin{tabbing}
					\hspace{2.5cm}\=\kill
					Betreuer: \>  \\
					\> Dozent im Fach \glqq \grqq \\
					\ \\
					Autor: \> Vorname Nachname \\
					\> Straße \\
					\> PLZ Ort \\
					\> Matrikel-Nr:  \\
					\> x. Fachsemester \\
					\ \\
					Ort, den \today
				\end{tabbing} 
			\end{center}
		\end{titlepage}

%\includepdf{Image/Deckblatt.pdf}

			\setcounter{tocdepth}{3}
			\setcounter{secnumdepth}{3}		
			\pagenumbering{Roman}
			\thispagestyle{empty}
			\pdfbookmark{\contentsname}{toc}\tableofcontents
			\newpage
			\listoffigures
			\listoftables
			\printglossary[type=acronym,style=listdotted,title=Abkürzungsverzeichnis,toctitle=Abkürzungsverzeichnis] 
			\newpage
			\pagenumbering{arabic}
			\thispagestyle{empty}

\chapter{Abstract}\label{abstract}
\ref{abstract}

\chapter{Introduction}\label{introduction}

\chapter{Related work}\label{related}
analysis of genetic mutations which cause breastcancer \autocite{breastcancer}

Zhao et al. \autocite{zhao_2016} describe how topic modeling can be used to analyze \gls{ngs}. By implementing topic modelling, text corpus are generated \autocite{zhao_2016}.

In the beginning of every genome analysis, there are several important questions to ask. Jurca et al. \autocite{jurca_2016} recommend to ask the following questions: What are the top studied genes in breast cancer? How regulated blood cancer research is in each country? Which countries have studied the largest number of breast cancer? Which are the popular genes mentioned together by countries every year? Where do key genes lie in the soft clusters?

Jurca et al. describe a process to use large-scale text analysis of biomedical abstracts in order to generate new hypothesis about cancer biomarkers \autocite{jurca_2016}. The target is to develop a data minng methodology that patterns in genes associated with cancer. By analyzing disease-specific gene expression data, experimental data is being checked whether a gene has indeed been upregulated or downregulated with respect to a disease.

\chapter{\gls{lda}}\label{lda}
\section{General description}\label{lda_description}
According to Jurca et al. \autocite{jurca_2016} the text mining process can be divided into four steps: First, the information has to be retrieved by user queries (\gls{ir}). Second, different vocabularies and ontologies have to be integrated (\gls{ner}). Third, during \gls{ie}, relationships between biological entities in the texts are extracted by either use co-occurence processing or \gls{nlp}. Last, there has to be gained biologically meaningful knowledge about how biological entities are related by using \gls{kd} methods.

Moreover, there can be distinguished between three types of clustering: hard clustering, hierarchical clustering and soft clusternig. Hard clustering describes the process of separating items into distinct groups where each item is exactly in one cluster. Hierarchical clustering implicates single-link (how similar the items are to one another) and complete-link (how dissimilar the items are). Soft clustering means that items cannot be distinctly separated into clusters and partly are member of two or more clusters at a time \autocite{jurca_2016}.  


\gls{lda} was developed by David Blei et. al in the year 2003 and is a clustering algorithm for text mining. It counts to the most popular topic modelling algorithms \autocite{zhao_2016}.
According to \autocite{zhao_2016}, topic modelling requires of a number documents which represent each of them a mixture of latent topics. Moreover, each topic is expressed by a distribution of words. During \gls{lda}, two relationships are analyzed: First, the relationship between documents and words, also called 'per-document topic distributions'. Second, the relationship between words and topics ('per-topic word distributions'). To measure the relationships exactly and to make inference about topics and documents for text mining, probability matrices are calculated.
 
\section{Examples and possible use cases}\label{lda_examples}

Zhao et al. describe the process of analyzing genomes as follows: First, each document corresponds to one of the total number of \gls{dna} straints. Second, all documents had the same number of words. Third, the distribution of words for topics as well as the distribution of topics in documents were described by random variables obeying Dirichlet distributions with parameters \alpha and \beta. After that, nucleotides and their orders in \gls{ngs} sequences coudl be treated as words and the genetic information in sequences was translated and exhibited as a 'bag of words' \autocite{zhao_2016}. By using the strain-topic matrix derived from topic modelling, relationships or similarities between the strains serotypes can be found out. 
 
images and schemes
\section{Python package 'Gensim'}\label{gensim}
              
\chapter{Acute Lymphoblastic Leukemia}\label{all}
\section{Types of Leukemia and its causes}\label{leukemia_types}
According to Jurca et al. \autocite{jurca_2016}, cancer is the result of damage, especially of mutations to cell's \gls{dna} which leads to a cell losing its normal functionality and gains the ability to indefinitely multiply until normal tissue funtions are impaired. This is also why malitious cancer is distributing so fast. Besides, each patient develops a different set of cancerous mutations in various genes which lead to multiple subtypes of cancer.
Furthermore, some genes can be up-regulated (which means that they are transcribed more and are expressed), down-regulated (which means that they are not expressed) or can be co-expressed (which means that they are expressed at the same time \autocite{jurca_2016}. 


\section{Examples for Genome Analysis: \gls{ngs}}\label{genome_analysis}
\gls{ngs} refers to post-Sanger sequencing methods \autocite{zhao_2016}. Furthermore, since \gls{ngs} produces large volumes of sequence data it might be very useful to use topic modelling techniques to maintain the flexibility for the level of resolution required for given experiments.  
The step before analyzing two or more (multiple) genomes is called alignment which includes a comparison of two genomes. There are many different types of alignments, but Zhao et al. refer to the \gls{msa} by describing \gls{muscle} and \gls{clustal}.

But there is not only the \gls{ngs} technique to analyze, but also many other methods to analyze genomes as described by \autocite{zhao_2016}. 


\section{Data sources: \gls{ncbi} and Ensembl genome browser 96}\label{datasources}

\chapter{Development of a solution for genetic analysis of \gls{all} genomes by implementing \gls{lda}}\label{development}
\section{Problems and challenges of genetic analysis}\label{problems_challenges}
\section{First steps: Draft of developed solution}\label{draft}

To get useful data, the \autocite{ncbi} was used to get all currently detected mutations of genomes which may cause \gls{lda}.

The first idea was to build a parsing application, which iterates over the found 582 genomes. After the iteration, it compares the oncogenes with the healthy genomes and to figure out where the differences are. The results might be displayed in a diagram. It might be possible to create clusters from the differences between the two groups or practice \gls{lda} on the differences.

\section{Proposed solution}\label{proposed_solution}
\section{Results}\label{results}

\chapter{Conclusion and Outlook}\label{conclusion_outlook}
\section{Lessons learned}\label{lessons_learned}
\section{Conclusion}\label{conclusion}
\section{Outlook}\label{outlook}	
	
% Beispiel für Bild
		\begin{figure}[htbp]
			\centering
			%\includegraphics[width=1\textwidth]{Image/xxx.pdf}
			\caption[xxx]{Bildunterschrift}
			\label{xxx}
		\end{figure}
		
		
% Beispiel Zitat		
		\begin{quote}
			\textit{Ein Zitat}
		\end{quote}


% Quellenangabe 		\autocite[Seite]{Autor.Jahr}
				\autocite[20]{}

\begingroup
	\setlength{\emergencystretch}{8em}
	\renewcommand{\bibname}{Literaturverzeichnis}
	\printbibliography
\endgroup
\newpage




\renewcommand{\bibname}{Rechtsquellenverzeichnis}

\makeatletter
\def\@biblabel#1{}
\makeatother


\begin{thebibliography}{0} 
%	\setlength{\leftmargin}{5cm}
	\setlength{\itemindent}{-0.2cm}
	\bibitem{} Bundesdatenschutzgesetz, BDSG, 1990, zuletzt geändert 2009
	\bibitem{} Gesetz gegen unlauteren Wettbewerb, UWG, 2004, zuletzt geändert 2013		
	\bibitem{} Richtlinie 95/46/EG des Europäischen Parlaments und des Rates vom 24. Oktober
	1995 NI. L 281/31 zum Schutz natürlicher Personen bei der Verarbeitung
	personenbezogener Daten und zum freien Datenverkehr
	\bibitem{} Telemediengesetz, TDG, 2007, zuletzt geändert 2010


\end{thebibliography}

\newpage

\ohead[]{Ehrenwörtliche Erklärung \hfill \thepage}

\null\vfill
\textbf{Ehrenwörtliche Erklärung}

Hiermit versichere ich, dass die vorliegende Arbeit von mir selbstständig und ohne unerlaubte Hilfe angefertigt worden ist, insbesondere dass ich alle Stellen, die wörtlich oder annähernd wörtlich aus Veröffentlichungen entnommen sind, durch Zitate als solche gekennzeichnet habe. Ich versichere auch, dass die von mir eingereichte schriftliche Version mit der digitalen Version übereinstimmt. Weiterhin erkläre ich, dass die Arbeit in gleicher oder ähnlicher Form noch keiner Prüfungsbehörde / Prüfungsstelle vorgelegen hat. Ich erkläre mich damit nicht einverstanden, dass die Arbeit der Öffentlichkeit zugänglich gemacht wird. Ich erkläre mich damit einverstanden, dass die Digitalversion dieser Arbeit zwecks Plagiatsprüfung auf die Server externer Anbieter hochgeladen werden darf. Die Plagiatsprüfung stellt keine Zurverfügungstellung für die Öffentlichkeit dar.

\ \\ \ \\ 


Ort, Datum (Vorname Nachname)

\vfill
\bibdata{Lit}



\end{document}
